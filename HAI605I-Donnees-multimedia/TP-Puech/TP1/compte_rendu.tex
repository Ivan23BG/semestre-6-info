\documentclass[french,a4paper,10pt]{article}

\usepackage[a4paper,hmargin=30mm,vmargin=30mm]{geometry}
\usepackage[T1]{fontenc} % font type
\usepackage[french]{babel} % language
\usepackage{lmodern} % font type
\usepackage[shortlabels]{enumitem}
\usepackage{hyperref}
\setlength{\parindent}{0pt}



\title{Compte Rendu TP1\\Prise en main d'une libraire de traitement d'images}
\author{Ivan Lejeune}
\date{\today}


\begin{document}
	\section{Seuillage d'une image au format pgm}\label{sec:1}

	\subsection{Ouverture des fichiers}\label{subsec:1.1}

	On commence par t\'el\'echarger les fichiers se trouvant \hyperref{https://www.lirmm.fr/~wpuech/enseignement/donnees_multimedia/librairie/}{ici}
	Ensuite, on les ouvre dans l'\'editeur de texte voulu, dans notre cas, on se servira de \emph{CLion} (et parfois de \emph{VSCodium}).
	
	\subsection{Téléchargement des fichiers nécessaires}\label{subsec:1.2}

	On t\'el\'echarge les fichiers depuis \hyperref{https://www.lirmm.fr/~wpuech/enseignement/donnees_multimedia/images/}{ici}.

	C'est parmi ces images qu'on effectuera la majorit\'e du travail.
\end{document}
