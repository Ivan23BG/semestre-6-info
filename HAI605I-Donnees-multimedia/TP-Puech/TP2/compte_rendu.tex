\documentclass[french,a4paper,10pt]{article}

\usepackage[a4paper,hmargin=30mm,vmargin=30mm]{geometry}
\usepackage[T1]{fontenc} % font type
\usepackage[french]{babel} % language
\usepackage{lmodern} % font type
\usepackage[shortlabels]{enumitem}
\usepackage{hyperref}
\usepackage{graphicx}
\usepackage{sectsty}
%\setlength{\parindent}{0pt}



\title{Compte Rendu TP2\\Opérations morphologies sur des images}
\author{Ivan Lejeune}
\date{\today}


\begin{document}
	\maketitle

	% make table of contents
	\tableofcontents

	\newpage
	\section{Seuillage d'une image et erosion de l'image binaire}\label{sec:1}

	\subsection{Choix de l'image}\label{subsec:1.1}

	On commence par choisir une image au format \emph{pgm}, dans notre cas, l'image \texttt{08.pgm}.
	On réduit ensuite la taille de l'image pour faciliter le traitement, on choisit une taille de 256x256 pixels.
	Cela donne alors :
	% insert original image and resized image
	\begin{figure}[!htb]
		\begin{minipage}{0.48\textwidth}
			\centering
			\fbox{\includegraphics[width=.7\linewidth]{./out/orig-08}}
			\caption{Image originale}\label{Fig:orig-08}
		\end{minipage}\hfill
		\begin{minipage}{0.48\textwidth}
			\centering
			\fbox{\includegraphics[width=.7\linewidth]{./out/resize-08}}
			\caption{Image redimensionnée}\label{Fig:resize-08}
		\end{minipage}
	\end{figure}

	\subsection{Seuillage de l'image}\label{subsec:1.2}

	On commence par modifier le programme \texttt{test\_grey.cpp} pour que les objets soient noirs et le fond blanc.
	On peut ensuite tester le seuillage de l'image avec différents seuils.
	Le plus pertinent semble être un seuil de 80, cela donne alors :
	% insert image and modified image
	\begin{figure}[!htb]
		\begin{minipage}{0.48\textwidth}
			\centering
			\fbox{\includegraphics[width=.7\linewidth]{./out/resize-08}}
			\caption{Image redimensionnée}\label{Fig:resize-08-2}
		\end{minipage}\hfill
		\begin{minipage}{0.48\textwidth}
			\centering
			\fbox{\includegraphics[width=.7\linewidth]{./out/test-grey-resize-08}}
			\caption{Image modifiée avec un seuil de 80}\label{Fig:test-grey-08}
		\end{minipage}
	\end{figure}

	\subsection{Erosion de l'image binaire}\label{subsec:1.3}

	On commence par créer un programme \texttt{erosion.cpp} pour effectuer l'érosion de l'image.
	Cela consiste à parcourir l'image et à remplacer chaque pixel par le maximum des pixels voisins.

	L'essentiel du code est le suivant : % insertion en image du code
	\begin{figure}[!htb]
		\centering
		\fbox{\includegraphics[width=.7\linewidth]{./out/erosion-code}}
		\caption{Code de l'érosion}\label{Fig:erosion-code}
	\end{figure}

	On peut ensuite tester l'érosion de l'image avec l'image seuillée précédemment.
	Cela donne alors :
	% insert image and modified image
	\begin{figure}[!htb]
		\begin{minipage}{0.48\textwidth}
			\centering
			\fbox{\includegraphics[width=.7\linewidth]{./out/test-grey-resize-08}}
			\caption{Image modifiée avec un seuil de 80}\label{Fig:test-grey-08-2}
		\end{minipage}\hfill
		\begin{minipage}{0.48\textwidth}
			\centering
			\fbox{\includegraphics[width=.7\linewidth]{./out/erosion-resize-08}}
			\caption{Image modifiée avec l'érosion}\label{Fig:erosion-test-grey-08}
		\end{minipage}
	\end{figure}

	\newpage
	\section{Seuillage d'une image et dilatation de l'image binaire}\label{sec:2}

	\subsection{Dilatation de l'image binaire}\label{subsec:2.1}

	On commence par créer un programme \texttt{dilatation.cpp} pour effectuer la dilatation de l'image.
	Cela consiste à parcourir l'image et à remplacer chaque pixel par le minimum des pixels voisins.

	L'essentiel du code est le suivant : % insertion en image du code
	\begin{figure}[!htb]
		\centering
		\fbox{\includegraphics[width=.7\linewidth]{./out/dilatation-code}}
		\caption{Code de la dilatation}\label{Fig:dilatation-code}
	\end{figure}

	On peut ensuite tester la dilatation de l'image avec l'image seuillée précédemment.
	Cela donne alors :
	% insert image and modified image
	\begin{figure}[!htb]
		\begin{minipage}{0.48\textwidth}
			\centering
			\fbox{\includegraphics[width=.7\linewidth]{./out/test-grey-resize-08}}
			\caption{Image modifiée avec un seuil de 80}\label{Fig:test-grey-08-3}
		\end{minipage}\hfill
		\begin{minipage}{0.48\textwidth}
			\centering
			\fbox{\includegraphics[width=.7\linewidth]{./out/dilatation-resize-08}}
			\caption{Image modifiée avec la dilatation}\label{Fig:dilatation-test-grey-08}
		\end{minipage}
	\end{figure}

	\newpage
	\section{Fermeture et ouverture d'une image et de l'image binaire}\label{sec:3}

	\subsection{Fermeture de l'image binaire}\label{subsec:3.1}

	On commence par créer un programme \texttt{fermeture.cpp} pour effectuer la fermeture de l'image.
	Cela consiste à effectuer une dilatation de l'image puis une érosion de l'image.
	Cela permet de fermer les trous dans les objets de l'image.

	L'essentiel du code est le suivant : % insertion en image du code
	\begin{figure}[!htb]
		\centering
		\fbox{\includegraphics[width=.7\linewidth]{./out/fermeture-code}}
		\caption{Code de la fermeture}\label{Fig:fermeture-code}
	\end{figure}

	L'essentiel du travail est de modifier les programmes précédents pour qu'ils puissent être utilisés dans ce
	programme.
	Pour cela, on crée des nouveaux fichiers sans \emph{main} et extrait le contenu d'\texttt{image\_ppm.h} dans un
	fichier \emph{cpp}.

	On peut ensuite tester la fermeture de l'image avec l'image seuillée précédemment.
	Cela donne alors :
	% insert image and modified image
	\begin{figure}[!htb]
		\begin{minipage}{0.48\textwidth}
			\centering
			\fbox{\includegraphics[width=.7\linewidth]{./out/test-grey-resize-08}}
			\caption{Image modifiée avec un seuil de 80}\label{Fig:test-grey-08-4}
		\end{minipage}\hfill
		\begin{minipage}{0.48\textwidth}
			\centering
			\fbox{\includegraphics[width=.7\linewidth]{./out/fermeture-resize-08}}
			\caption{Image modifiée avec la fermeture}\label{Fig:fermeture-test-grey-08}
		\end{minipage}
	\end{figure}

	\subsection{Ouverture de l'image binaire}\label{subsec:3.2}

	On commence par créer un programme \texttt{ouverture.cpp} pour effectuer l'ouverture de l'image.
	Cela consiste à effectuer une érosion de l'image puis une dilatation de l'image.
	Cela permet de supprimer les petits objets de l'image.

	L'essentiel du code est le suivant : % insertion en image du code
	\begin{figure}[!htb]
		\centering
		\fbox{\includegraphics[width=.7\linewidth]{./out/ouverture-code}}
		\caption{Code de l'ouverture}\label{Fig:ouverture-code}
	\end{figure}

	On peut ensuite tester l'ouverture de l'image avec l'image seuillée précédemment.
	Cela donne alors :
	% insert image and modified image
	\begin{figure}[!htb]
		\begin{minipage}{0.48\textwidth}
			\centering
			\fbox{\includegraphics[width=.7\linewidth]{./out/test-grey-resize-08}}
			\caption{Image modifiée avec un seuil de 80}\label{Fig:test-grey-08-5}
		\end{minipage}\hfill
		\begin{minipage}{0.48\textwidth}
			\centering
			\fbox{\includegraphics[width=.7\linewidth]{./out/ouverture-resize-08}}
			\caption{Image modifiée avec l'ouverture}\label{Fig:ouverture-test-grey-08}
		\end{minipage}
	\end{figure}

	\subsection{Enchainement de fermeture et ouverture}\label{subsec:3.3}

	On peut ensuite enchaîner la fermeture et l'ouverture de l'image pour obtenir une image plus propre.
	Cela donne alors :
	% insert original then closed then reopened image
	\begin{figure}[!htb]
		\begin{minipage}{0.30\textwidth}
			\centering
			\fbox{\includegraphics[width=.7\linewidth]{./out/test-grey-resize-08}}
			\caption{Image modifiée avec un seuil de 80}\label{Fig:test-grey-08-6}
		\end{minipage}\hfill
		\begin{minipage}{0.30\textwidth}
			\centering
			\fbox{\includegraphics[width=.7\linewidth]{./out/fermeture-resize-08}}
			\caption{Image modifiée avec fermeture}\label{Fig:fermeture-ouverture-test-grey-08}
		\end{minipage}
		\begin{minipage}{0.30\textwidth}
			\centering
			\fbox{\includegraphics[width=.7\linewidth]{./out/ouverture-fermeture-resize-08}}
			\caption{Image modifiée avec fermeture puis ouverture}\label{Fig:ouverture-fermeture-test-grey-08}
		\end{minipage}
	\end{figure}

	\subsection{Impact cumulatif de la fermeture et de l'ouverture}\label{subsec:3.4}

	On peut ensuite tester l'impact cumulatif de la fermeture et de l'ouverture de l'image.
	On procède d'abord à 3 dilatations, 6 érosions et enfin 3 dilatations.

	Cela donne alors :
	% insert original then modified image
	\begin{figure}[!htb]
		\begin{minipage}{0.48\textwidth}
			\centering
			\fbox{\includegraphics[width=.7\linewidth]{./out/test-grey-resize-08}}
			\caption{Image modifiée avec un seuil de 80}\label{Fig:test-grey-08-7}
		\end{minipage}\hfill
		\begin{minipage}{0.48\textwidth}
			\centering
			\fbox{\includegraphics[width=.7\linewidth]{./out/edil3-resize-08}}
			\caption{Image modifiée avec cumul}\label{Fig:fermeture-ouverture-test-grey-08-2}
		\end{minipage}
	\end{figure}

\end{document}
