\documentclass[french,a4paper,10pt]{article}
\makeatletter
%--------------------------------------------------------------------------------
\usepackage[T1]{fontenc} % font type
\usepackage[french]{babel} % language
\usepackage{lmodern} % font type
\usepackage[shortlabels]{enumitem}
\setlist[itemize,1]{label={\color{gray}\small \textbullet}} % customises itemize default -
\usepackage{fancyhdr} % customises head and foot-notes
\usepackage{centernot} % allows centering \not with \centernot
\usepackage{stmaryrd} % allows \llbracket

\usepackage{xcolor} % colour customisation, extends to tables with {colortbl}
\definecolor{astral}{RGB}{46,116,181}
\definecolor{verdant}{RGB}{96,172,128}
\definecolor{calculus-coral}{RGB}{255,191,191} % exercice colour
\definecolor{matrix-mist}{RGB}{204,204,204} % remark colour
\definecolor{quadratic-quartz}{RGB}{204,153,153} % example colour 


\usepackage{latexsym}
\usepackage{amsmath}
\usepackage{amsfonts}
\usepackage{amssymb}
\usepackage{amsthm}

\usepackage[ruled,vlined]{algorithm2e}




\newtheoremstyle{gen-style}{\topsep}{\topsep}%
{}%         Body font
{}%         Indent amount (empty = no indent, \parindent = para indent)
{\sffamily\bfseries}% Thm head font
{.}%        Punctuation after thm head
{ }%     Space after thm head (\newline = linebreak)
{\thmname{#1}\thmnumber{~#2}\thmnote{~#3}}%         Thm head spec


\newtheoremstyle{no-num-style}{\topsep}{\topsep}%
{}%         Body font
{}%         Indent amount (empty = no indent, \parindent = para indent)
{\sffamily\bfseries}% Thm head font
{.}%        Punctuation after thm head
{ }%     Space after thm head (\newline = linebreak)
{\thmname{#1}\thmnote{~#3}}%         Thm head spec


\usepackage[]{mdframed}

\newcommand{\mytheorem}[5]{%
	\ifstrequal{#5}{o}{%
		\newmdtheoremenv[
		hidealllines=true,
		leftline=true,
		skipabove=0pt,
		innertopmargin=-5pt,
		innerbottommargin=2pt,
		linewidth=4pt,
		innerrightmargin=0pt,
		linecolor=#3,
		]{#1}[#4]{#2}%
	}{%
		\newmdtheoremenv[
		hidealllines=true,
		leftline=true,
		skipabove=0pt,
		innertopmargin=-5pt,
		innerbottommargin=2pt,
		linewidth=4pt,
		innerrightmargin=0pt,
		linecolor=#3,
		]{#1}{#2}[#4]%
	}%
}

\theoremstyle{gen-style}
\mytheorem{example}{Exemple}{quadratic-quartz}{section}{}
\mytheorem{remark}{Remarque}{matrix-mist}{section}{}

\theoremstyle{no-num-style}
\mytheorem{td-sol}{Solution}{verdant}{}{}
\mytheorem{rappel}{Rappel}{matrix-mist}{section}{}
\mytheorem{td-exo}{Exercice}{calculus-coral}{}{}

%---------------
% Mise en page
%--------------

\setlength{\parindent}{0pt}

\providecommand{\defemph}[1]{{\sffamily\bfseries\color{astral}#1}}


\usepackage{sectsty}
\allsectionsfont{\color{astral}\normalfont\sffamily\bfseries}

\usepackage{mathrsfs}

%----- Easy way to redeclare math operators -----
\makeatletter
\newcommand\RedeclareMathOperator{%
	\@ifstar{\def\rmo@s{m}\rmo@redeclare}{\def\rmo@s{o}\rmo@redeclare}%
}
\newcommand\rmo@redeclare[2]{%
	\begingroup \escapechar\m@ne\xdef\@gtempa{{\string#1}}\endgroup
	\expandafter\@ifundefined\@gtempa
	{\@latex@error{\noexpand#1undefined}\@ehc}%
	\relax
	\expandafter\rmo@declmathop\rmo@s{#1}{#2}}
\newcommand\rmo@declmathop[3]{%
	\DeclareRobustCommand{#2}{\qopname\newmcodes@#1{#3}}%
}
\@onlypreamble\RedeclareMathOperator
\makeatother

\newcommand{\skipline}{\vspace{\baselineskip}}
\newcommand{\noi}{\noindent}
%------------------------------------------------


\newcommand{\ol}[1]{\overline{#1}} % overline
\newcommand{\ul}[1]{\underline{#1}} % underline
\newcommand{\sub}{\subset} % subset
\newcommand{\scr}[1]{\mathscr{#1}} % scr rapide
\newcommand{\bb}[1]{\mathbb{#1}} % bb rapide
\newcommand{\restr}[2]{#1\mathop{}\!|_{#2}} % restriction


%----- Intervalles -----
\newcommand{\oo}[1]{\mathopen{]}#1\mathclose{[}}
\newcommand{\of}[1]{\mathopen{]}#1\mathclose{]}}
\newcommand{\fo}[1]{\mathopen{[}#1\mathclose{[}}
\newcommand{\ff}[1]{\mathopen{[}#1\mathclose{]}}



\makeatother

\usepackage[a4paper,hmargin=30mm,vmargin=30mm]{geometry}
\title{\color{astral} \sffamily \bfseries TD de préparation à l'examen 1}
\author{Ivan Lejeune\thanks{Feuille inspirée de M. Giroudeau}}
\date{\today}

\begin{document}
	
	\maketitle
	
    %ex1
	\begin{td-exo}[1]
		Soient $f$ et $g$ deux fonctions calculables. Considérons
        \[
            E=\{x\mid f(x)=0\text{ ou }g(x)=0\}
        \]
        \begin{enumerate}
            \item Donner un algorithme qui affiche les éléments de $E$.
            \item Que peut-on en déduire sur $E$.
        \end{enumerate}
	\end{td-exo}

    %sol
    \begin{td-sol}\,
        \begin{enumerate}
            \item On peut écrire un algorithme qui affiche les éléments de $E$ en utilisant la
            fonction de temps $h$. Cela donne l'algorithme suivant:\\
                \begin{algorithm}[H]
                    \caption{Affichage des éléments de $E$}
                    \Entree{$\forall x,\forall t$}
                    \Sortie{Affiche $x$ si $f(x)=0$ ou $g(x)=0$}
                    \Si{$(h(f,x,t)\land f(x)=0) \lor (h(y,x,t)\land f(x)=0)$}{
                        Afficher $x$
                    }
                \end{algorithm}
            \item On peut en déduire que $E$ est récursivement énumérable.
        \end{enumerate}
    \end{td-sol}

    %ex2
    \begin{td-exo}[2]
        Soit $f$ une bijection des suites finies d'entiers dans $\bb N$ définie 
        $f(x_1,x_2,\dots,x_k)$ où la liste est $(x_1,x_2,\dots,x_k)$ et $x_i\in\bb N$.
        En déduire une fonction $g$ bijective des suites croissantes (au sens large
        i.e. $x_i\leq x_{i+1}$) finies d'entiers dans $\bb N$.
    \end{td-exo}
	
    %sol


    %ex3
    \begin{td-exo}[3]\,
        \begin{enumerate}
            \item Soient $f$ et $g$ deux fonctions calculables. 
            Soit $E=\{x\mid f(x)$ est défini, $g(x)$ est défini et $f(x)<g(x)\}$. 
            Montrer que $E$ est récursivement énumérable.

            \item Le complémentaire de $E$ est-il récursivement énumérable? Justifiez complètement votre réponse.

            \item Soit $p$ une procédure définie pour tout $x$. 
            Montrer que savoir si $\forall x,p(x)$ est premier est un problème indécidable.

            \item Soit une suite $f_i$ de fonctions totales de $\bb N$ dans $\bb N$. 
            Donner une fonction croissante $h$ qui n'appartienne pas à cette suite.
        \end{enumerate}
    \end{td-exo}

    %sol
    \begin{td-sol}\,
        \begin{enumerate}
            \item On peut écrire un algorithme qui affiche les éléments de $E$ en utilisant la
            fonction de temps $h$. Cela donne l'algorithme suivant:\\
                \begin{algorithm}[H]
                    \caption{Affichage des éléments de $E$}
                    \Entree{$\forall x,\forall t$}
                    \Sortie{Affiche $x$ si $f(x)$ est défini, $g(x)$ est défini et $f(x)<g(x)$}
                    \Si{$h(f,x,t)\land h(g,x,t)\land f(x)<g(x)$}{
                        Afficher $x$
                    }
                \end{algorithm}
            \item Le complémentaire de $E$ n'est pas récursivement énumérable. 
            En effet, si le complémentaire de $E$ était récursivement énumérable, 
            alors $E$ serait décidable, ce qui n'est pas forcément le cas.
            \item Soit $P$ le prédicat suivant:
            \[
                P(p) = \begin{cases}
                    1& \text{si }\forall x, p(x) \text{ est premier}\\
                    0 & \text{sinon}
                \end{cases}
            \]

            On considère
            \[
                \text{Int } p_0(n) \mapsto n \quad \text{et} \quad \text{Int } p_1(n) \mapsto 3
            \]
            Alors
            \[
                P(p_0) = 1 \quad \text{et} \quad P(p_1) = 0
            \]
            Donc $P$ n'est pas trivial.
            Alors d'après le théorème de Rice, $P$ est indécidable.


            \item On considère la fonction $h$ suivante:
            \[
                h(i)=\sum_{k=0}^i f_k(i)+i
            \]
            Alors $h$ est croissante et n'appartient pas à la suite $f_i$.
        \end{enumerate}
    \end{td-sol}

    %ex4
    \begin{td-exo}[4 - Réduction en $K$ et $K_0$]\,\\
        Soit $K = \{x\mid x\in \bb N,\phi_x(x)\downarrow\}$ et $K_0 = \{(x,y)\mid x\in \bb N, y\in \bb N, \phi_x(y)\downarrow\}$.
        
        On rappelle que la notation $\phi_x(y)\downarrow$ (resp. $\phi_x(y)\uparrow$) signifie que la fonction 
        récursive $\phi_x$ converge (resp. diverge) en $y$.
        Montrer que $K_0$ est indécidable.
    \end{td-exo}

    %sol
    \begin{td-sol}
        Il suffit de voir que $K$ est un cas particulier de $K_0$.
        Alors $K\subseteq K_0$.
        Or $K$ est indécidable (argument diagonal de Cantor).

        Donc $K_0$ est indécidable.
    \end{td-sol}

    %ex5
    \begin{td-exo}[5 - Un nouvel ensemble $A$]\,\\
        Soit $A=\{x\mid x\in \bb N,\phi_x$ est une fonction constante $\}$.
        Montrer que $A$ est indécidable.
        Pour cela, considérons la fonction
        \[
            g(x,y) = \begin{cases}
                0 & \text{si } \phi_x(x) \downarrow\\
                \uparrow & \text{sinon}
            \end{cases}
        \]
        Est-ce que $g$ est calculable?
    \end{td-exo}

    %sol
    \begin{td-sol}
        Supposons que $x\in K$. Alors $\phi_x(x)\downarrow$. Donc $g(x,y)=0$.

        Supposons que $x\notin K$. Alors $\phi_x(x)\uparrow$. Donc $g$ n'est pas constante.

        Donc $A$ est indécidable.
    \end{td-sol}

    %ex6
    \begin{td-exo}[6 - Un nouvel ensemble $B$]\,\\
        Soit $A=\{x\mid x\in \bb N,\phi_x(4)=12\}$.
        Montrer que $B$ est indécidable.

        Pour cela, considérons la fonction
        \[
            g(x,y) = \begin{cases}
                12 & \text{si } \phi_x(x)\downarrow\\
                \uparrow & \text{sinon}
            \end{cases}
        \]

        Est-ce que $g$ est calculable?
    \end{td-exo}

    %sol
    \begin{td-sol}
        Supposons que $x\in K$. Alors $\phi_x(x)\downarrow$. Donc $g(x,y)=12$.

        Supposons que $x\notin K$. Alors $\phi_x(x)\uparrow$. Donc $g$ n'est pas constante.

        Donc $B$ est indécidable.
    \end{td-sol}

    %ex7
    \begin{td-exo}[7 - Un nouvel ensemble $C$]\,\\
        Soit $C=\{x\mid x\in \bb N,\phi_x(23)\uparrow\}$.
        Montrer que $C$ est indécidable.
    \end{td-exo}

    %sol
    \begin{td-sol}
        Supposons que $x\in K$. Alors $\phi_x(23)\downarrow$. Donc $x\notin C$.

        Supposons que $x\notin K$. Alors $\phi_x(23)\uparrow$. Donc $x\in C$.

        Donc $C$ est indécidable.
    \end{td-sol}

    %ex8
    \begin{td-exo}[8]
        Dans les cas suivants, donner un exemple d'ensemble $A\subseteq \bb N$ ou montrer qu'il n'en existe pas:
        \begin{enumerate}
            \item $A$ est récursif et $\ol A$ est récursif.
            \item $A$ est récursif et $\ol A$ n'est pas récursif.
            \item $A$ est récursif et $\ol A$ est récursivement énumérable.
            \item $A$ est récursif et $\ol A$ n'est pas récursivement énumérable.
            \item $A$ est récursivement énumérable et $\ol A$ n'est pas récursif.
            \item $A$ est récursivement énumérable et $\ol A$ est récursivement énumérable.
            \item $A$ est récursivement énumérable et $\ol A$ n'est pas récursivement énumérable.
            \item $A$ n'est pas récursivement énumérable et $\ol A$ n'est pas récursivement énumérable.
        \end{enumerate}
    \end{td-exo}

    %sol
    \begin{td-sol}\,
        \begin{enumerate}
            \item On peut prendre $A=\bb N$. Alors $A$ est récursif et $\ol A=\emptyset$ est récursif.
            \item On ne peut pas trouver un tel ensemble $A$.
            \item On peut prendre $A=\bb N$. Alors $A$ est récursif et $\ol A=\emptyset$ est récursivement énumérable.
            \item On ne peut pas trouver un tel ensemble $A$.
            \item Voir exercice 3.
            \item On peut prendre $A=\bb N$. Alors $A$ est récursivement énumérable et $\ol A=\emptyset$ est récursivement énumérable.
            \item Voir exercice 3.
            \item Voir le problème de l'arrêt.
        \end{enumerate}
    \end{td-sol}

\end{document}



